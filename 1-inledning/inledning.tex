% !TeX spellcheck = sv_SE
\section{Inledning}

I den här rapporten vill jag presentera en lösning för att komma åt och föra in sånger på internet. Systemet kommer bestå av en vanlig webbsida som innehåller alla sånger, samt en funktion för att föra in nya sånger. Fokus kommer ligga på att göra webbsidan användbar för mobilanvändare. Målet är att \textsc{utn}:s \cite{utn} och alla sektioners sånger skall finnas med, såsom datavetare \cite{manualen} och matematiker \cite{moebius} som har sina egna sångböcker. Webbsidan kommer designas för att i första han användas av mobiltelefoner, d.v.s. för att användas av små skärmar. För personer med nedsättningar är det viktigt att gällande standarder (\textsc{aria} \cite{aria}) följs så att deras hjälpmedel kan tolka webbsidan korrekt.

Det är inte ovanligt att sångböcker glöms till fester, eller att sånger sjungs som inte finns med i alla sångböcker. Dessutom finns det varianter av vissa sånger (exempelvis \emph{Måsen}, s. 138 i \cite{utn}, s. 86 i \cite{manualen} och nr. 20\nicefrac{1}{2} i \cite{moebius}). En elektronisk sångbok skulle enkelt kunna utökas för nya sånger och erbjuda alla varianter. Min sångbok kommer främst rikta in sig på Uppsalastudenter på TekNat-fakulteten, så \emph{de facto} sångboken, \emph{Uppsala teknolog- och naturvetarkårs sångbok} \cite{utn}, kommer ligga till grund för ett webbaserat dito. Den senaste av \textsc{utn}:s sångböcker trycktes i 8000 exemplar 2012. Detta motsvarar ca. 0,8 miljoner A5-ark, eller ca. 330 träd enligt \emph{Treeproject} \cite{treeproject}. En övergång till elektroniska sångböcker skulle därför ge en positiv inverkan på användandet av papper.

Många sånger är allmängods, men många är också nya texter till existerande melodier eller helt enkelt existerande sånger. Detta kräver enligt \emph{Musikförläggarna} \cite{musikf} tillstånd från upphovsrättsägarna. Andra hävdar att kultur inte skall påverkas av marknadsekonomiska krafter för att istället spridas fritt. \cite{vp}

%I nuläget finns de flesta sånger som sjungs av studenter i form av tryckta sångböcker. Eftersom utbudet av sånger är stort är böckerna antingen otympliga eller bristande. Dessutom bidrar de till användningen av papper. En elektronisk sångbok kommer inte använda papper, och användarvänligheten kommer inte påverkas av antalet sånger. Det finns dock vissa aspekter som en elektronisk sångbok inte kan ersätta och många föredrar därför fysiska böcker. De behöver inget batteri, går utmärkt att läsa i starkt solljus och erbjuder våra ögon en naturligare upplevelse.

%Den elektroniska sångboken utnyttjar populariteten hos \emph{smartphones} för att göra sångerna mer tillgängliga. Böcker kan glömmas, men mobilen är nästan alltid med. Det kommer även vara mycket lättare att hitta rätt sång, både genom sökning av titel, kategori och innehåll samt länkar från sånger till andra. Sångboken siktar även på att vara komplett. Detta skall underlättas med en lättanvänd webapp för att enkelt föra in nya sånger.