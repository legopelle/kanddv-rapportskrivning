% !TeX spellcheck = sv_SE
\section{Slutsats}

En sångbok på nätet skulle lösa många av de problemen som en fysisk sångbok har. Både miljö- och användaraspekter kan förbättras med en elektronisk sångbok. Genom att använda våra oumbärliga mobiler kan sångboken och dess sånger nå många fler till ett lägre pris. Eftersom en webbaserad sångbok kan uppdateras kontinuerligt behöver den inte bytas ut, och kan förbli aktuell. Utöver det kan den lätt erbjuda varianter av sånger utan problem, något som inte görs i tryckta böcker, och därför ge en större gemenskap och tillhörighet mellan alla TekNat:s sektioner.

Genom att använda moderna webbteknologier som responsiv design och \textsc{aria}, kan webbsidan göras tillgänglig för nästan alla, även de med nedsättningar. Vissa problem kan dock ses med gällande lagstiftning för sångtexter, och vi måste vara noggranna för att få tillstånd att publicera vissa sånger.

I framtiden kan den även utökas för hela Sveriges studentsånger. Den nuvarande implementationen sätter inga gränser för det, utan det är bara en fråga om arbete. Det kan även göras lättare att föra in nya sånger genom ett grafiskt gränssnitt. Då bör dock lagringen av sångerna övergå från Git-versioner till en databas. För att optimera upplevelsen på mobiler kan det även göras app.:ar för Android, i\textsc{os}, Sailfish, m.m.